%| report template for CSUS Senior Design
%|
%| language: LaTeX
%| Author: Ben Smith
%| 
%| This source has been tagged with the "#CHANGE" tag in areas
%| that require updating when making a new docuent
%|
%| This source will generate a PDF file complete with thumbnails navigation menu and metadata,
%| 

\documentclass[12pt,article]{IEEEtran}

%| This section will essentially create variables to be used in some
%| of the documents formatting and the PDF's metadata 
%|
%| 
\newcommand{\TITLE}{Assignment Title}
\newcommand{\KEYWORDS}{keyword1, keyword2, keyword1, keyword2, keyword1, keyword2, keyword1, keyword2}
\newcommand{\ABSTRACT}{The average person is becoming less active, not just in first world countries, 
					but across geopolitical and socioeconomic boundaries. This sedentary lifestyle has dire consequences 
					to health.  Studies have shown that even moderate exercise leads to improved health across all age
					groups.  We analyze some of the benefits, and challenges, of the alternative commuter.}
\newcommand{\AUTHOR}{Micheal Frith, David Larribas, Devin Moore, Benjamin Smith}
\newcommand{\DUEDATE}{September 10, 2013}
%| =================================================================================================
%| Formatting options
%| =================================================================================================

%| Enables PDF metadata, thumbnails, and navigation
\newcommand\MYhyperrefoptions{
	bookmarks=true,
	bookmarksnumbered=true,
	bookmarksopen=true,
	bookmarkstype={toc},
	pdfpagemode={UseOutlines},
	plainpages=false,
	pdfpagelabels=true,
	colorlinks=true,
	linkcolor={black},
	citecolor={black},
	pagecolor={black},
	urlcolor={black},
	final=true,
	pdftex,
	pdftitle={\TITLE},															%#CHANGE
	pdfsubject={\ABSTRACT},														%#CHANGE
	pdfauthor={Micheal Frith, David Larribas, Devin Moore, Benjamin Smith},		%#CHANGE
	pdfkeywords={\KEYWORDS}}													%#CHANGE

%| Calls hyper ref package 
\usepackage[\MYhyperrefoptions]{hyperref}

%| IEEE Citation package
\usepackage{cite}

%| American Mathematical Society package for fancy maths	
\usepackage[cmex10]{amsmath}
\interdisplaylinepenalty=2500				% Restores IEEE line spacing after amsmath

%| Better tables than LaTeX 2e
\usepackage{array}

%| Improved URL handling
\usepackage{url}

\usepackage{graphicx}

\begin{document}

	%| Inserts header cover sheet 
	\begin{titlepage}
	\begin{center}
		\vspace{20 cm}
		
		\textsc{\LARGE EEE190: Senior Design}\\[1.3cm]
		
		\textsc{\Large \DUEDATE}\\[0.5cm]
		
		\vspace{5 mm}
		
		% Title
		\rule{415pt}{2pt}\\
		{ \huge \bfseries \TITLE \\[0.2cm] }
		\rule{415pt}{2pt}\\
		
		\vspace{10mm}
		
		%| Author names
		\begin{minipage}{0.4\textwidth}
		\begin{flushleft} \large
		
		\emph{Authors:}\\
			Micheal 		\textsc{frith}\\
			David 			\textsc{Larribas}\\
			Devin 			\textsc{moore}\\
			Benjamin		\textsc{smith}\\
		\end{flushleft}
		\end{minipage}
		\begin{minipage}{0.4\textwidth}
		\begin{flushright} \large
		
		%| Faculty names
		\emph{Supervisors:} \\
			Fethi 	\textsc{Belkhouche} \\
			Russ	\textsc{Tatro}
		\end{flushright}
		\end{minipage}
	\end{center}
	
	%| gives the names a bit of breathing room
	\vspace{30mm}
	
	\begin{center}
	\begin{minipage}{420pt}
		%| Automatic abstract entry from main document
		\begin{flushleft} \large
			\begin{abstract}
				\ABSTRACT \\
			\end{abstract}
		\end{flushleft}
		
		%| Automatic keyword entry from main document
		\begin{flushleft} \large
			\begin{keywords}
				\KEYWORDS \\
			\end{keywords}
		\end{flushleft}
	\end{minipage}
	
	%| Fill the remainder of the page
	\vfill
		
	% Bottom of the page sac state logo
	\begin{center}
		\includegraphics[width=0.2\textwidth]{./logo}~\\[1cm]
	\end{center}
\end{titlepage}
		 
	%| =================================================================================================
	%| Main Document body begins here!(the writin' parts)
	%| =================================================================================================
	\section{Introduction}

	\section{The problem's reach}
		\IEEEPARstart{I}{nactivity} Inactivity and its health effects were once the exclusive blight of 
		the developed world. This problem has changed its face over the past few decades to impact an 
		increasing percentage of the world'€™s populace. The World Health Organization  
		\cite{18} 
	\section{The problem's reach}
	Inactivity carries with it major detrimental effects to human health including physical and mental afflictions.  The physical and mental handicaps include stress, depression, obesity and heart disease.  These afflictions carry with them a reduced life span, inflated health insurance costs and a lower quality of life.
	Obesity is an elevated topic of discussion because of the great number of detrimental effects it has on the human body.  Each year there are 2.8 million deaths worldwide caused by being overweight or obese.  35\% of all adults twenty years of age or older were overweight, with many them being obese, according to a survey in 2008. \cite{14}  Obesity and inactivity can cause cardiovascular disease, high cholesterol, diabetes, different types of cancers and psychological ailments.  Nearly half of all adults worldwide are affected by cardiovascular disease which contributes to stroke and kidney failure.  There were an estimated 17.3 million deaths caused by cardiovascular disease accounting for 30\% of all deaths worldwide making it the number one cause of death. \cite{4}  Hypertension alone, a form of cardiovascular disease, is responsible for 9.4 million deaths per year worldwide. \cite{3}  Diabetes doubles the risk of death compared to peers of the same age.  Diabetes will damage the heart, kidney, nerves and even blood vessels. \cite{15}
	Inactivity can cause all the previous ailments, but physical activity can alleviate or even cure these effects. Physical activity increases energy levels with as little as 30 minutes of exercise a day. \cite{16, p.805} This can promote many benefits at home and in the workplace such as increased energy and facilitate daily tasks.  Adequate physical activity can provide energy throughout the day which can increase focus on any task, physical or not. \cite{17, p1447}  Regular physical activity will increase cardiorespiratory and muscular fitness. \cite{3} Individuals can increase healthiness by engaging in regular physical activity 150 minutes a week for adults, which is less than 30 minutes per day. \cite{18}
	In addition to the physical conditions caused by physical inactivity, it also causes many negative mental effects.  These include depression, anxiety and low self-esteem.  Depression can cause a decrease in work performance which results in a significant decrease in employee profitability. \cite{11, p.12}  More than 350 million people worldwide have depression making it one of the leading causes of disabilities in the world.  \cite{12} Anxiety can be beneficial in small doses, but a large amount of anxiety can turn into a hindering disorder. These disorders include generalized anxiety disorder, obsessive-compulsive disorder, panic disorder and posttraumatic stress disorder. \cite{13}  
	With the increase in weight control that comes with physical activity, a better self-image may also be produced.  Increased physical activity can cause a person to look and feel more in shape.  Physical activity can lead to an increase in performance and has been shown to lead to improved self-esteem.  Physical activity can also relieve anxiety. \cite{13} Depression will also improve with moderate levels of physical activity. \cite{16, p.806}
	As one ages, moderate aerobic and strength training activities 3-5 times a week for 30-60 minutes can improve aspects of mental health, such as improved thinking, learning and judgment abilities. These activities can also help with movement, keeping joints, bones, and muscles healthy as well as slowing bone density loss. \cite{18}

	\section{The problem's reach}

\bibliographystyle{IEEEtran}

\bibliography{IEEEfull}

\end{document}