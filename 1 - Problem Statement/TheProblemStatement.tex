%<<<<<<< HEAD
%| report template for CSUS Senior Design
%|
%| language: LaTeX
%| Author: Ben Smith
%| 
%| This source has been tagged with the "#CHANGE" tag in areas
%| that require updating when making a new docuent
%|
%| This source will generate a PDF file complete with thumbnails navigation menu and metadata,
%| 

\documentclass[12pt,article]{IEEEtran}

%| This section will essentially create variables to be used in some
%| of the documents formatting and the PDF's metadata 
%|s
%| 
\newcommand{\TITLE}{The Problem Statement}
\newcommand{\KEYWORDS}{keyword1, keyword2, keyword1, keyword2, keyword1, keyword2, keyword1, keyword2}
\newcommand{\ABSTRACT}{The majority of society is becoming less active, which is leading to an increase in health problems. This problem is not specific to first world societies, although they tend to have more options for becoming active. Most people claim they don't have enough time to go out and get exercise, so we aim to allow people to get exercise while going about their daily tasks. With assisting people in getting active, we believe it is help steer society away from the sedentary lifestyle.}
\newcommand{\AUTHOR}{Micheal Frith, David Larribas, Devin Moore, Benjamin Smith}
\newcommand{\DUEDATE}{September 10, 2013}
%| =================================================================================================
%| Formatting options
%| =================================================================================================

%| Enables PDF metadata, thumbnails, and navigation
\newcommand\MYhyperrefoptions{
	bookmarks=true,
	bookmarksnumbered=true,
	bookmarksopen=true,
	bookmarkstype={toc},
	pdfpagemode={UseOutlines},
	plainpages=false,
	pdfpagelabels=true,
	colorlinks=true,
	linkcolor={black},
	citecolor={black},
	pagecolor={black},
	urlcolor={black},
	final=true,
	pdftex,
	pdftitle={\TITLE},															%#CHANGE
	pdfsubject={\ABSTRACT},														%#CHANGE
	pdfauthor={Micheal Frith, David Larribas, Devin Moore, Benjamin Smith},					%#CHANGE
	pdfkeywords={\KEYWORDS}}													%#CHANGE

%| Calls hyper ref package 
\usepackage[\MYhyperrefoptions]{hyperref}
%| Override compsoc class' Palatino font for body text, restores to Times New Roman
\renewcommand{\rmdefault}{ptm}\selectfont

%| IEEE Citation package
\usepackage{cite}

%| American Mathematical Society package for fancy maths	
\usepackage[cmex10]{amsmath}
\interdisplaylinepenalty=2500				% Restores IEEE line spacing after amsmath

%| Better tables than LaTeX 2e
\usepackage{array}

%| Improved URL handling
\usepackage{url}

% correct bad hyphenation here
\hyphenation{op-tical net-works semi-conduc-tor}

\usepackage{graphicx}

\begin{document}

%| Inserts header cover sheet 
\begin{titlepage}
	\begin{center}
		\vspace{20 cm}
		
		\textsc{\LARGE EEE190: Senior Design}\\[1.3cm]
		
		\textsc{\Large \DUEDATE}\\[0.5cm]
		
		\vspace{5 mm}
		
		% Title
		\rule{415pt}{2pt}\\
		{ \huge \bfseries \TITLE \\[0.2cm] }
		\rule{415pt}{2pt}\\
		
		\vspace{10mm}
		
		%| Author names
		\begin{minipage}{0.4\textwidth}
		\begin{flushleft} \large
		
		\emph{Authors:}\\
			Micheal 		\textsc{frith}\\
			David 			\textsc{Larribas}\\
			Devin 			\textsc{moore}\\
			Benjamin		\textsc{smith}\\
		\end{flushleft}
		\end{minipage}
		\begin{minipage}{0.4\textwidth}
		\begin{flushright} \large
		
		%| Faculty names
		\emph{Supervisors:} \\
			Fethi 	\textsc{Belkhouche} \\
			Russ	\textsc{Tatro}
		\end{flushright}
		\end{minipage}
	\end{center}
	
	%| gives the names a bit of breathing room
	\vspace{30mm}
	
	\begin{center}
	\begin{minipage}{420pt}
		%| Automatic abstract entry from main document
		\begin{flushleft} \large
			\begin{abstract}
				\ABSTRACT \\
			\end{abstract}
		\end{flushleft}
		
		%| Automatic keyword entry from main document
		\begin{flushleft} \large
			\begin{keywords}
				\KEYWORDS \\
			\end{keywords}
		\end{flushleft}
	\end{minipage}
	
	%| Fill the remainder of the page
	\vfill
		
	% Bottom of the page sac state logo
	\begin{center}
		\includegraphics[width=0.2\textwidth]{./logo}~\\[1cm]
	\end{center}
\end{titlepage}
	 
%| =================================================================================================
%| Main Document body begins here!(the writin' parts)
%| =================================================================================================
\section{Introduction}
	\subsection{Group meeting minutes and synopsis of past week}
	
	Lorem ipsum dolor sit amet, consectetur adipiscing elit. Cras sollicitudin, sapien vel gravida gravida, ligula magna ullamcorper orci, eu sodales nunc metus ac ante. Curabitur vel fermentum sem, sit amet congue velit. Maecenas cursus vel orci nec varius. Fusce a aliquet eros. Nulla facilisi. Duis auctor, tellus a suscipit porta, sem turpis sagittis arcu, et dictum neque erat vitae justo. Nunc scelerisque vehicula ipsum, non condimentum nisi cursus et. Integer nulla neque, consectetur quis pellentesque eget, pulvinar id tellus. Donec fermentum turpis ut leo posuere, at dignissim mauris accumsan. Donec ut enim a nibh elementum varius. Curabitur tincidunt rutrum elit, et aliquam eros sagittis aliquet. In mattis, diam eu euismod vehicula, lorem augue congue quam, et fringilla lectus sem sed tellus.
	
	%| Summary of team activities
	\begin{table}[ht]
		\renewcommand{\arraystretch}{1.3}
		\caption{Summary of team activities}
		
		\label{Summary of team activities}
		
		\centering
		\begin{tabular}{p{5.5cm}|p{1cm}|p{1cm}}
		\hline
		\bfseries 	Task		 		& \bfseries Hours Worked	& \bfseries Status	\\
		\hline
		\hline
					Michael Frith		& 2.0						& 20\%				\\	%| Percentage signs are escaped by a \ to indicate they are not a LaTeX command
					David Larribas 		& 2.0						& 20\%				\\	
					Devin Moore 		& 2.0						& 20\%				\\	
					Benjamin Smith		& 2.0						& 9001\%			\\	
		\hline
		\end{tabular}
		
		\subsection{TOC TEST}
		
	\end{table}

\section{Inactivity}
\PARstart{I}{nactivity}Inactivity is becoming an increasingly large problem all around the world. Human beings must get regular exercise to keep their bodies in shape and free of sickness. It is becoming even more apparent, with an increasing amount of research being done across the globe, that physical activity also helps keep the mind healthy. This problem of people being inactive is not restricted to any specific socio economic sector.

Since the the advent of horse drawn carriages, humans have been finding easier ways to travel which require less physical work. Now that most modern societies contain acces to cars, taxis, busses, subways and the like, It’s becoming much easier to get to where you need to be without having to walk much further than the length of your driveway.  As generations of people became so used to getting around by motorized vehicles, even children came to see this as the best solution to travel. People of all ages around the world are using motorized transportation of sometype. They could be driving their own car, riding a bus, an elderly care van, or being driven to school by their parents. The reasons for this behavior can range greatly. Some people see it as safer because walking in certain areas can have different dangers associated with it at certain times of the day. People in a modernized society especially, simply don’t have time to walk the distance required to get to work, school, stores, and friends houses. 

Leisure used to mean going outside, playing some sort of game with friends or family. Maybe some swimming, hiking, something of that sort. Increasingly it means sitting in front of a Television, computer monitor, or cellphone. [[We can get tooonnnns of statistics here]]]. This is more prevalent in developed nations such as the US, however, technology is increasing worldwide at a substantial rate. Many times when people want to go out and get away from activities like that, they simply go to a movie theater, see a play, or eat at a restaurant. Much of our modern leisure is sedentary.

A large percentage of modern jobs require people to sit in one position for the majority of the day. [[[Insert Penile Statistics HERE]]]. Desk jobs, cubicles, computer work, and meeting all involve sitting in one spot for extended periods of time.

School is completely necessary, but again, requires students to simply sit in the same position for extended periods of time. During the earlier years of school, there is usualy a small amount of time set asside for Physical Educatin. However people can be waived from this for all sorts of reasons, even for being too out of shape[[[A statistic here would be nice]]]]]. The people that are able to get PE waived are, a lot of the time, the ones who could benifit from it the most. 

Sleep, school, work, leisure, and going in between all of those activities, generally make up the majority of peoples’ day. It is very common that modern humans try to pack in as much as they can as efficiently as they can, so it is very rare to be able to devote more than an hour to some outside cause like exersize. But what if there was a way to fit in exersize with only adding a few incriments of 10-15 minutes throughout your day? How would that benifit a human, and how can it be done?

\section{Health Effects}
 \PARstart{I}{nactivity} carries with it major detrimental effects to human health including physical and mental afflictions.  The physical and mental handicaps include stress, depression, obesity and heart disease.  These afflictions carry with them a reduced life span, inflated health insurance costs and a lower quality of life.

Obesity is an elevated topic of discussion because of the great number of detrimental effects it has on the human body.  Each year there are 2.8 million deaths worldwide caused by being overweight or obese.  35\% of all adults twenty years of age or older were overweight, with a large number of them being obese, according to a survey in 2008. (WHO Obesity).  Obesity and inactivity can cause cardiovascular disease, high cholesterol, diabetes, different types of cancers and a number of psychological ailments.  Nearly half of all adults worldwide are affected by cardiovascular disease which  contributes to stroke and kidney failure.  There were an estimated 17.3 million deaths caused by cardiovascular disease accounting for 30\% of all deaths worldwide making it the number one cause of death.  Hypertension alone, a form of cardiovascular disease, is responsible for 9.4 million deaths per year worldwide.  Diabetes doubles the risk of death compared to peers of the same age.  Diabetes will damage the heart, kidney, nerves and even blood vessels.
Not only does inactivity aid in causing the previous physical effects, physical activity can alleviate or even cure these effects entirely. Physical activity increases energy levels with as little as 30 minutes of exertion a day.  This can promote many benefits at home and in the workplace such as increased energy and facilitate daily tasks.  Adequate physical activity coupled with sleep can provide energy throughout the day which can increase focus on any task, physical or not.  Regular physical activity will increase cardiorespiratory and muscular fitness.

\section{Options}
\PARstart{T}{here} are many different options for indivuals to combat the adverse health effects of a sedentary lifestyle. From medications that suppress symptoms, to extreme diets that aleviate certain problems. However, both of those solutions can potentially create other health problems of their own. The best way to avoid health problems stemming from inactivity, is simply to become more active. Recent studies suggest that at least 30 minutes of activity per day can [[[statistics]]]]. This exercise can come from something as simple as walking.
Going to the gym, swimming, playing physical sports, walking

\end{document}

\end{document}
=======
\documentclass[12pt,journal,compsoc]{IEEEtran}

\providecommand{\PSforPDF}[1]{#1}

\newcommand\MYhyperrefoptions{bookmarks=true,bookmarksnumbered=true,
	pdfpagemode={UseOutlines},plainpages=false,pdfpagelabels=true,
	colorlinks=true,linkcolor={black},citecolor={black},pagecolor={black},
	urlcolor={black},
	pdftitle={Bare Demo of IEEEtran.cls for Computer Society Journals},
	pdfsubject={Typesetting},
	pdfauthor={Michael D. Shell},
	pdfkeywords={Computer Society, IEEEtran, journal, LaTeX, paper, template}}
	             
% correct bad hyphenation here
\hyphenation{op-tical net-works semi-conduc-tor}


\begin{document}
%
% paper title
\title{Problem Statement: Inactivity and it's Adverse Effects on Health\\Team 3 Senior Project }

\author{Michael~Frith,
       David~Larribas,        
       Devin~Moore,
       and~Benjamin~Smith% <-this % stops a space
\IEEEcompsocitemizethanks{\IEEEcompsocthanksitem The Department
of Electrical and Computer Engineering, Sacramento State University, Sacramento, CA, 95819.\protect\\} %

% note need leading \protect in front of \\ to get a newline within \thanks as
% \\ is fragile and will error, could use \hfil\break instead.
%E-mail: see http://www.michaelshell.org/contact.html
%\IEEEcompsocthanksitem J. Doe and J. Doe are with Anonymous University.}% <-this % stops a space
%\thanks{Manuscript received April 19, 2005; revised January 11, 2007.}
}

% note the % following the last \IEEEmembership and also \thanks - 
% these prevent an unwanted space from occurring between the last author name
% and the end of the author line. i.e., if you had this:
% 
% \author{....lastname \thanks{...} \thanks{...} }
%                     ^------------^------------^----Do not want these spaces!
%
% a space would be appended to the last name and could cause every name on that
% line to be shifted left slightly. This is one of those "LaTeX things". For
% instance, "\textbf{A} \textbf{B}" will typeset as "A B" not "AB". To get
% "AB" then you have to do: "\textbf{A}\textbf{B}"
% \thanks is no different in this regard, so shield the last } of each \thanks
% that ends a line with a % and do not let a space in before the next \thanks.
% Spaces after \IEEEmembership other than the last one are OK (and needed) as
% you are supposed to have spaces between the names. For what it is worth,
% this is a minor point as most people would not even notice if the said evil
% space somehow managed to creep in.



% The paper headers
\markboth{Team 3 Societal Problem Description,~Vol.~1,~NO.~1,~\today.}%
{Shell \MakeLowercase{\textit{et al.}}: Bare Advanced Demo of IEEEtran.cls for Journals} 
% The only time the second header will appear is for the odd numbered pages
% after the title page when using the twoside option.
% 
% *** Note that you probably will NOT want to include the author's ***
% *** name in the headers of peer review papers.                   ***
% You can use \ifCLASSOPTIONpeerreview for conditional compilation here if
% you desire.



% The publisher's ID mark at the bottom of the page is less important with
% Computer Society journal papers as those publications place the marks
% outside of the main text columns and, therefore, unlike regular IEEE
% journals, the available text space is not reduced by their presence.
% If you want to put a publisher's ID mark on the page you can do it like
% this:
%\IEEEpubid{0000--0000/00\$00.00~\copyright~2007 IEEE}
% or like this to get the Computer Society new two part style.
%\IEEEpubid{\makebox[\columnwidth]{\hfill 0000--0000/00/\$00.00~\copyright~2007 IEEE}%
%\hspace{\columnsep}\makebox[\columnwidth]{Published by the IEEE Computer Society\hfill}}
% Remember, if you use this you must call \IEEEpubidadjcol in the second
% column for its text to clear the IEEEpubid mark (Computer Society jorunal
% papers don't need this extra clearance.)



% use for special paper notices
%\IEEEspecialpapernotice{(Invited Paper)}



% for Computer Society papers, we must declare the abstract and index terms
% PRIOR to the title within the \IEEEcompsoctitleabstractindextext IEEEtran
% command as these need to go into the title area created by \maketitle.
\IEEEcompsoctitleabstractindextext{%
\begin{abstract}
%\boldmath
The majority of society is becoming less active, which is leading to an increase in health problems. This problem is not specific to first world societies, although they tend to have more options for becoming active. Most people claim they don't have enough time to go out and get exercise, so we aim to allow people to get exercise while going about their daily tasks. With assisting people in getting active, we believe it is help steer society away from the sedentary lifestyle. 
\end{abstract}
% IEEEtran.cls defaults to using nonbold math in the Abstract.
% This preserves the distinction between vectors and scalars. However,
% if the journal you are submitting to favors bold math in the abstract,
% then you can use LaTeX's standard command \boldmath at the very start
% of the abstract to achieve this. Many IEEE journals frown on math
% in the abstract anyway. In particular, the Computer Society does
% not want either math or citations to appear in the abstract.

% Note that keywords are not normally used for peerreview papers.
\begin{IEEEkeywords}
Computer Society, IEEEtran, journal, \LaTeX, paper, template.
\end{IEEEkeywords}}


% make the title area
\maketitle


% To allow for easy dual compilation without having to reenter the
% abstract/keywords data, the \IEEEcompsoctitleabstractindextext text will
% not be used in maketitle, but will appear (i.e., to be "transported")
% here as \IEEEdisplaynotcompsoctitleabstractindextext when compsoc mode
% is not selected <OR> if conference mode is selected - because compsoc
% conference papers position the abstract like regular (non-compsoc)
% papers do!
\IEEEdisplaynotcompsoctitleabstractindextext
% \IEEEdisplaynotcompsoctitleabstractindextext has no effect when using
% compsoc under a non-conference mode.


% For peer review papers, you can put extra information on the cover
% page as needed:
% \ifCLASSOPTIONpeerreview
% \begin{center} \bfseries EDICS Category: 3-BBND \end{center}
% \fi
%
% For peerreview papers, this IEEEtran command inserts a page break and
% creates the second title. It will be ignored for other modes.
\IEEEpeerreviewmaketitle



\section{Introduction}
% Computer Society journal papers do something a tad strange with the very
% first section heading (almost always called "Introduction"). They place it
% ABOVE the main text! IEEEtran.cls currently does not do this for you.
% However, You can achieve this effect by making LaTeX jump through some
% hoops via something like:
%
%\ifCLASSOPTIONcompsoc
%  \noindent\raisebox{2\baselineskip}[0pt][0pt]%
%  {\parbox{\columnwidth}{\section{Introduction}\label{sec:introduction}%
%  \global\everypar=\everypar}}%
%  \vspace{-1\baselineskip}\vspace{-\parskip}\par
%\else
%  \section{Introduction}\label{sec:introduction}\par
%\fi
%
% Admittedly, this is a hack and may well be fragile, but seems to do the
% trick for me. Note the need to keep any \label that may be used right
% after \section in the above as the hack puts \section within a raised box.



% The very first letter is a 2 line initial drop letter followed
% by the rest of the first word in caps (small caps for compsoc).
% 
% form to use if the first word consists of a single letter:
% \IEEEPARstart{A}{demo} file is ....
% 
% form to use if you need the single drop letter followed by
% normal text (unknown if ever used by IEEE):
% \IEEEPARstart{A}{}demo file is ....
% 
% Some journals put the first two words in caps:
% \IEEEPARstart{T}{his demo} file is ....
% 
% Here we have the typical use of a "T" for an initial drop letter
% and "HIS" in caps to complete the first word.
\IEEEPARstart{T}{his} demo file is intended to serve as a ``starter file''
for IEEE Computer Society journal papers produced under \LaTeX\ using
IEEEtran.cls version 1.7 and later.
% You must have at least 2 lines in the paragraph with the drop letter
% (should never be an issue)
I wish you the best of success.

\hfill mds
 
\hfill January 11, 2007

\subsection{Inactivity}
Subsection text here.

% needed in second column of first page if using \IEEEpubid
%\IEEEpubidadjcol

\subsubsection{Ages}
Subsubsection text here.

\subsubsection{Geographic}

\subsubsection{Physical Handicaps}

\subsection{Health Effects}

\subsubsection{Physical}
Obesity, Cardio vascular
\subsubsection{Mental}
Stress, Axiety, Depression

\subsection{Options}
Gym, Swimming, Running, Walking, Bikes

\subsection{E-Bike}
\subsubsection{Why}
\subsubsection{Whats Been Done}

% An example of a floating figure using the graphicx package.
% Note that \label must occur AFTER (or within) \caption.
% For figures, \caption should occur after the \includegraphics.
% Note that IEEEtran v1.7 and later has special internal code that
% is designed to preserve the operation of \label within \caption
% even when the captionsoff option is in effect. However, because
% of issues like this, it may be the safest practice to put all your
% \label just after \caption rather than within \caption{}.
%
% Reminder: the "draftcls" or "draftclsnofoot", not "draft", class
% option should be used if it is desired that the figures are to be
% displayed while in draft mode.
%
%\begin{figure}[!t]
%\centering
%\includegraphics[width=2.5in]{myfigure}
% where an .eps filename suffix will be assumed under latex, 
% and a .pdf suffix will be assumed for pdflatex; or what has been declared
% via \DeclareGraphicsExtensions.
%\caption{Simulation Results}
%\label{fig_sim}
%\end{figure}

% Note that IEEE typically puts floats only at the top, even when this
% results in a large percentage of a column being occupied by floats.
% However, the Computer Society has been known to put floats at the bottom.


% An example of a double column floating figure using two subfigures.
% (The subfig.sty package must be loaded for this to work.)
% The subfigure \label commands are set within each subfloat command, the
% \label for the overall figure must come after \caption.
% \hfil must be used as a separator to get equal spacing.
% The subfigure.sty package works much the same way, except \subfigure is
% used instead of \subfloat.
%
%\begin{figure*}[!t]
%\centerline{\subfloat[Case I]\includegraphics[width=2.5in]{subfigcase1}%
%\label{fig_first_case}}
%\hfil
%\subfloat[Case II]{\includegraphics[width=2.5in]{subfigcase2}%
%\label{fig_second_case}}}
%\caption{Simulation results}
%\label{fig_sim}
%\end{figure*}
%
% Note that often IEEE papers with subfigures do not employ subfigure
% captions (using the optional argument to \subfloat), but instead will
% reference/describe all of them (a), (b), etc., within the main caption.


% An example of a floating table. Note that, for IEEE style tables, the 
% \caption command should come BEFORE the table. Table text will default to
% \footnotesize as IEEE normally uses this smaller font for tables.
% The \label must come after \caption as always.
%
%\begin{table}[!t]
%% increase table row spacing, adjust to taste
%\renewcommand{\arraystretch}{1.3}
% if using array.sty, it might be a good idea to tweak the value of
% \extrarowheight as needed to properly center the text within the cells
%\caption{An Example of a Table}
%\label{table_example}
%\centering
%% Some packages, such as MDW tools, offer better commands for making tables
%% than the plain LaTeX2e tabular which is used here.
%\begin{tabular}{|c||c|}
%\hline
%One & Two\\
%\hline
%Three & Four\\
%\hline
%\end{tabular}
%\end{table}


% Note that IEEE does not put floats in the very first column - or typically
% anywhere on the first page for that matter. Also, in-text middle ("here")
% positioning is not used. Most IEEE journals use top floats exclusively.
% However, Computer Society journals sometimes do use bottom floats - bear
% this in mind when choosing appropriate optional arguments for the
% figure/table environments.
% Note that, LaTeX2e, unlike IEEE journals, places footnotes above bottom
% floats. This can be corrected via the \fnbelowfloat command of the
% stfloats package.



\section{Conclusion}
The conclusion goes here.





% if have a single appendix:
%\appendix[Proof of the Zonklar Equations]
% or
%\appendix  % for no appendix heading
% do not use \section anymore after \appendix, only \section*
% is possibly needed

% use appendices with more than one appendix
% then use \section to start each appendix
% you must declare a \section before using any
% \subsection or using \label (\appendices by itself
% starts a section numbered zero.)
%


\appendices
\section{Proof of the First Zonklar Equation}
Appendix one text goes here.

% you can choose not to have a title for an appendix
% if you want by leaving the argument blank
\section{}
Appendix two text goes here.


% use section* for acknowledgement
\ifCLASSOPTIONcompsoc
  % The Computer Society usually uses the plural form
  \section*{Acknowledgments}
\else
  % regular IEEE prefers the singular form
  \section*{Acknowledgment}
\fi


The authors would like to thank...


% Can use something like this to put references on a page
% by themselves when using endfloat and the captionsoff option.
\ifCLASSOPTIONcaptionsoff
  \newpage
\fi



% trigger a \newpage just before the given reference
% number - used to balance the columns on the last page
% adjust value as needed - may need to be readjusted if
% the document is modified later
%\IEEEtriggeratref{8}
% The "triggered" command can be changed if desired:
%\IEEEtriggercmd{\enlargethispage{-5in}}

% references section

% can use a bibliography generated by BibTeX as a .bbl file
% BibTeX documentation can be easily obtained at:
% http://www.ctan.org/tex-archive/biblio/bibtex/contrib/doc/
% The IEEEtran BibTeX style support page is at:
% http://www.michaelshell.org/tex/ieeetran/bibtex/
%\bibliographystyle{IEEEtran}
% argument is your BibTeX string definitions and bibliography database(s)
%\bibliography{IEEEabrv,../bib/paper}
%
% <OR> manually copy in the resultant .bbl file
% set second argument of \begin to the number of references
% (used to reserve space for the reference number labels box)
\begin{thebibliography}{1}

\bibitem{IEEEhowto:kopka}
H.~Kopka and P.~W. Daly, \emph{A Guide to {\LaTeX}}, 3rd~ed.\hskip 1em plus
  0.5em minus 0.4em\relax Harlow, England: Addison-Wesley, 1999.

\end{thebibliography}

% biography section
% 
% If you have an EPS/PDF photo (graphicx package needed) extra braces are
% needed around the contents of the optional argument to biography to prevent
% the LaTeX parser from getting confused when it sees the complicated
% \includegraphics command within an optional argument. (You could create
% your own custom macro containing the \includegraphics command to make things
% simpler here.)
%\begin{biography}[{\includegraphics[width=1in,height=1.25in,clip,keepaspectratio]{mshell}}]{Michael Shell}
% or if you just want to reserve a space for a photo:

\begin{IEEEbiography}{Michael Shell}
Biography text here.
\end{IEEEbiography}

% if you will not have a photo at all:
\begin{IEEEbiographynophoto}{John Doe}
Biography text here.
\end{IEEEbiographynophoto}

% insert where needed to balance the two columns on the last page with
% biographies
%\newpage

\begin{IEEEbiographynophoto}{Jane Doe}
Biography text here.
\end{IEEEbiographynophoto}

% You can push biographies down or up by placing
% a \vfill before or after them. The appropriate
% use of \vfill depends on what kind of text is
% on the last page and whether or not the columns
% are being equalized.

%\vfill

% Can be used to pull up biographies so that the bottom of the last one
% is flush with the other column.
%\enlargethispage{-5in}



% that's all folks
\end{document}


>>>>>>> refs/remotes/origin/ProblemStatement
